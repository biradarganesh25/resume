\documentclass{article}

% \usepackage[top=0.25in, bottom=0.25in, left=0.5in, right=0.5in]{geometry}
% \usepackage{geometry}
\usepackage{geometry}
 \geometry{
 a4paper,
 total={170mm,257mm},
 left=12mm,
 right=12mm,
 top=13mm,
 bottom=5mm,
 footskip=5mm
 }
\usepackage{enumitem}
\usepackage{tabto}

\usepackage[hidelinks]{hyperref}

\pagenumbering{gobble}
\usepackage{lastpage}

\usepackage{fancyhdr}
\pagestyle{fancy}

\hypersetup{
 colorlinks=false,
    linkcolor=blue,
    filecolor=magenta,      
    urlcolor=cyan,
    pdftitle={Resume}
}

\renewcommand{\headrulewidth}{0pt} % to remove line on header
\renewcommand{\footrulewidth}{0pt} % to remove line on footer

\begin{document}
\begin{center}
% \thispagestyle{empty}
\large \textbf{Ganeshprasad Biradar \\}
\normalsize biradarganesh25@gmail.com $\mid$ 9797212892 $\mid$ \href{https://www.linkedin.com/in/biradarganesh25}{\underline{linkedin.com/in/biradarganesh25}} $\mid$ \href{https://github.com/biradarganesh25}{\underline{github.com/biradarganesh25}} \\
\hrulefill
\end{center}


%%%%%%%%%%%%%%%%%%%%%%%%%%%%%%%%%%%%%%%%%%%%%%%%%%%%%%%%%%%%%%%
% OBJECTIVE
% Who you are, what domain, what are you looking for and when?
%%%%%%%%%%%%%%%%%%%%%%%%%%%%%%%%%%%%%%%%%%%%%%%%%%%%%%%%%%%%%%%

%%%%%%%%%%%%%%%%%%%%%%%%%%%%%%%%%%%%%%%%%%%%%%%%%%%%%%%%%%%%%%%
% SKILLS: Important and relevant to the job you are applying for
%%%%%%%%%%%%%%%%%%%%%%%%%%%%%%%%%%%%%%%%%%%%%%%%%%%%%%%%%%%%%%%

% \noindent \textbf{\underline{CORE SKILLS}} \\
% Skill 1 (years of experience), Skill 2 (years of experience), Skill 3 (years of experience) \\
% Skill 1 (level of expertise), Skill 2 (level of expertise), Skill 3 (level of expertise) \\

%%%%%%%%%%%%%%%%%%%%%%%%%%%%%%%%%%%%%%%%%%%%%%%%%%%%%%%%%%%%%%%
% EDUCATION
% University name, degree, year of graduation, GPA (optional)
%%%%%%%%%%%%%%%%%%%%%%%%%%%%%%%%%%%%%%%%%%%%%%%%%%%%%%%%%%%%%%%
\noindent \textbf{\underline{EDUCATION}} \\
\textbf{Texas A\&M University} \hfill College Station, Texas, USA \\
\textit{Master Of Science in Computer Science}  \tabto*{9cm} GPA: 4.00/4 \hfill August 2021 $-$ May 2023 \\
\textit{Relevant Coursework: Distributed Systems and Cloud Computing, Analysis of Algorithms, Software Engineering}\\

\noindent \textbf{R.V. College of Engineering} \hfill Bangalore, Karnataka, India \\
\textit{Bachelor of Engineering in Computer Science} \tabto*{9cm} GPA: 9.09/10 \hfill August 2015 $-$ August 2019 \\

%%%%%%%%%%%%%%%%%%%%%%%%%%%%%%%%%%%%%%%%%%%%%%%%%%%%%%%%%%%%%%%
% WORK EXPERIENCE
% What did you do? -> Project goals OR what problem did you solve?
% How did you do it? -> Skills and technologies
% What impact did you create? -> Numbers and percentages.
% Example: 
% + Developed an app for matching mentor and mentees for Android and iOS platform.
% + Successfully matched 85% of the applications and randomized the rest.
% 
% Talk about team work, initiative, soft skills.
%
% Can also include personal projects, competitions, contribution to Open source.
%%%%%%%%%%%%%%%%%%%%%%%%%%%%%%%%%%%%%%%%%%%%%%%%%%%%%%%%%%%%%%%
\noindent \textbf{\underline{WORK EXPERIENCE}} \\
\noindent \textbf{Cockroach Labs} \hfill New York City, New York, USA\\
\textit{Software Engineering Intern} \hfill Sept 2022 $-$ Dec 2022
\begin{itemize}[noitemsep,nolistsep,leftmargin=*]
	\item Explored different approaches to encode \href{https://www.cockroachlabs.com/docs/stable/change-data-capture-overview.html}{\underline{changefeeds}} from CockroachDB in Apache Parquet format and leveraged existing distributed changefeed processors to add Parquet support to CDC in CockroachDB. This involved understanding different layers of distributed database (key-value storage, distributed transactions, SQL).
	\item Designed and implemented a new mechanism for scheduling changefeeds at regular intervals. This along with support for Parquet format will enable users to use changefeeds as an alternative to exporting data out of CockroachDB, with better fault tolerance and resiliency. I also designed the SQL statement grammar for creating schedule changefeeds, which involved brushing up on different topics like CFG (Context-free grammar), Shift-reduce parser, using Yacc and Lex to build parsers, etc.
	\item Skills: Distributed databases and computing, Go, go-yacc, Automata theory. \\
\end{itemize}

\noindent \textbf{NVIDIA} \hfill Santa Clara, California, USA\\
\textit{Software Engineering Intern} \hfill May 2022 $-$ August 2022
\begin{itemize}[noitemsep,nolistsep,leftmargin=*]
	\item Researched different rate limiting designs for Unified Access Management (UAM) service (which will handle authorization for all Nvidia's cloud services) and implemented a global rate limiting solution which used Lyft's open sourced ratelimit service and Redis. This was a critical requirement for scaling the UAM service. 
	\item Analyzed response time of the rate limiting service and optimized connection parameters to the service to reduce latency and satisfy SLA's. Final latency: p99 was 15ms.
	\item Designed and implemented APIs to dynamically change the global ratelimit configuration with zero downtime with 95\% unit test coverage.
	\item Read Designing Data-Intensive Applications book, which helped me understand intricacies of distributed systems and make better design choices (for e.g. choices involving trade-offs between availability and consistency).
	\item Skills: Go, go-kit, testify, Envoy, AWS, Kubernetes, Redis, Terraform, Cassandra, Distributed Systems. \\
\end{itemize}

\noindent \textbf{Texas A\&M University} \hfill College Station, Texas, USA\\
\textit{Graduate Research Assistant (Conversational AI Developer) - Soft Research Lab} \hfill January 2022 $-$ May 2022
\begin{itemize}[noitemsep,nolistsep,leftmargin=*]
\item {Led a team of 2 to add conversational intelligence to VR characters in different VR applications like \href{https://softinteraction.com/portfolio/sbirt-vr}{\underline{SBIRT-VR}}}.
\item {Developed, deployed and maintained a conversational ML model which exhibited different patient personalities that students used to practice their diagnostic skills.}
\item Skills: Rasa, Python, Flask, Gunicorn, PyTorch, BERT and its variants, Scikit-Learn, NLP.\\

\end{itemize}
\noindent \textbf{Citrix R\&D} \hfill Bangalore, Karnataka, India \\
\textit{Software Development Engineer 2} \hfill April 2021 $-$ August 2021 \\
\textit{Software Development Engineer} \hfill July 2019 $-$ March 2021 \\
\textit{Software Development Intern} \hfill January 2019 $-$ June 2019
\begin{itemize}[noitemsep,nolistsep,leftmargin=*]
\item {Designed and implemented cloud modules for Azure and GCP for High Availability and Backend Autoscaling features in Citrix's Application Delivery Controller (ADC). These features increased the adoption of the product by around 20\%.}
\item {Proposed and led an initiative to create a unified API for communicating with different cloud providers that made development of new features of Citrix ADC 3 times faster on these clouds.}
\item {Recognized that existing library used for testing Citrix ADC (built in-house) was not suitable to test cloud specific features (only black box testing was being done). Started a new initiative to add unit tests by mocking cloud services.}
\item {Developed a new framework to automate the deployment of complex network configurations using Citrix ADC on different clouds like GCP, AWS, and Azure, which reduced the average testing time for new releases of the product by around 60\%. Won `Highest Revenue Impact' and `Most Popular Choice' awards at Citrix's Techfair (2019) for this project.}
\item {Skills: AWS, Azure, GCP, Python, C, Ansible.\\}
\end{itemize}

\pagebreak

\noindent \textbf{Samsung R\&D} \hfill Bangalore, Karnataka, India \\
\textit{Software Development Intern} \hfill June 2018 $-$ August 2018
\begin{itemize}[noitemsep,nolistsep,leftmargin=*]
\item {Co-led a team of 4 to create an ML model by experimenting with different unsupervised clustering algorithms like DBSCAN, K-MEANS, and OPTICS to identify under-performing eNodeBs in a highly overloaded LTE network using Key Performance Indicators (KPIs), which entirely eliminated manual supervision to detect congested eNodeBs.}
\item{Skills: R, DBSCAN, K-MEANS, OPTICS.\\}
\end{itemize}

%%%%%%%%%%%%%%%%%%%%%%%%%%%%%%%%%%%%%%%%%%%%%%%%%%%%%%%%%%%%%%
% Other Skills: you can add all your other skills here.
% Continue to keep only relevant skills
%%%%%%%%%%%%%%%%%%%%%%%%%%%%%%%%%%%%%%%%%%%%%%%%%%%%%%%%%%%%%%%
\noindent \textbf{\underline{TECHNICAL SKILLS AND OPEN SOURCE CONTRIBUTIONS}}
\begin{itemize}[noitemsep,nolistsep,leftmargin=*]
\item {Programming Languages: C, C++, Go, SQL, Python, R}
\item {Technologies:  Kubernetes, Envoy, AWS, GCP, Azure, Redis, Terraform, Ansible, Numpy, Pandas, PyTorch, Seaborn, CUDA, OpenMP, Docker, Spark, Rasa, go-kit, testify, go-yacc}
\item {Other: Distributed Systems, Microservices, Network Programming in C, Concurrency, Multithreading and Synchronization, Automata Theory}
\item {Open Source Contributions:}
\begin{itemize}
    \item CockroachDB: It is a distributed, strongly consistent SQL database written in Go. I have contributed to their SQL parsing layer. 
    \item Zulip: It is an open source chat and collaborative software. I have contributed to their user interface.
\end{itemize}
\end{itemize}

%%%%%%%%%%%%%%%%%%%%%%%%%%%%%%%%%%%%%%%%%%%%%%%%%%%%%%%%%%%%%%
% Accomplishments
%%%%%%%%%%%%%%%%%%%%%%%%%%%%%%%%%%%%%%%%%%%%%%%%%%%%%%%%%%%%%%
%%%%%%%%%%%%%%%%%%%%%%%%%%%%%%%%%%%%%%%%%%%%%%%%%%%%%%%%%%%%%%%
% PROJECT
% What did you do?
% How did you do it? -> Skills and technologies
% What impact did you create? -> Numbers and percentages.
%
% Talk about team work, initiative, soft skills.
%
% Can also include personal projects, competitions, contribution to Open source.
%%%%%%%%%%%%%%%%%%%%%%%%%%%%%%%%%%%%%%%%%%%%%%%%%%%%%%%%%%%%%%%


%%%%%%%%%%%%%%%%%%%%%%%%%%%%%%%%%%%%%%%%%%%%%%%% %%%%%%%%%%%%%%%
% Extra Curricular Activities, Leadership, etc 
%%%%%%%%%%%%%%%%%%%%%%%%%%%%%%%%%%%%%%%%%%%%%%%%%%%%%%%%%%%%%%%
%\cfoot{{Page \thepage\ of \pageref{LastPage}}}


\end{document}
